\section{Исследование, связанное с географическими координатами}


Попробуем увидеть разницу между войной и битвой по анализу этого свойства "coordinate location".

\textbf{Обозначения.} Будем говорить, что объект обладает точной геопривязкой, если у него есть заполненное свойство "coordinate location" с указанием долготы и широты места.

\textbf{Гипотеза.} У войн будет меньше указано координат, поскольку война - это обычно что-то более протяжённое во времени и пространстве.

Cписок битв с заполненными данными в поле "coordinate location":

\begin{itemize}
    \item Объект: \href{https://www.wikidata.org/wiki/Q6256}{страна (Q6256)},
    \item Свойство: \href{https://www.wikidata.org/wiki/Property:P571}{дата основания (P571)}.
\end{itemize}

\begin{lstlisting}[language=SPARQL]
#List of `instances of` "countries without an inception" 
SELECT ?country ?countryLabel 
WHERE
{
    ?country wdt:P31 wd:Q6256.       #country
    
    MINUS { ?country wdt:P571 [] } . #inception of country
    SERVICE wikibase:label { bd:serviceParam wikibase:language "en" }
}
\end{lstlisting}

\href{https://query.wikidata.org/#%23List%20of%20%60instances%20of%60%20%22countries%20without%20a%20inception%22%20%0ASELECT%20%3Fcountry%20%3FcountryLabel%20%0AWHERE%0A%7B%0A%20%20%20%20%3Fcountry%20wdt%3AP31%20wd%3AQ6256.%20%23country%0A%20%20%20%20%0A%20%20%20%20MINUS%20%7B%20%3Fcountry%20wdt%3AP571%20%5B%5D%20%7D%20.%20%23inception%20of%20country%0A%20%20%20%20SERVICE%20wikibase%3Alabel%20%7B%20bd%3AserviceParam%20wikibase%3Alanguage%20%22en%22%20%7D%0A%7D%0A%0A}{SPARQL-запрос}, 100 записей.

Итак, на 6 марта 2017 года Викиданные содержат 100 из 198 записей о ныне существующих странах с неизвестным годом основания страны.
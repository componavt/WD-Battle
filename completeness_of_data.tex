\section{География военных конфликтов}


Попробуем увидеть разницу между войной и битвой по анализу этого свойства "coordinate location".

\textbf{Обозначения.} Будем говорить, что объект обладает точной геопривязкой, если у него есть заполненное свойство "coordinate location" с указанием долготы и широты места.

Объект обладает геопривязкой, если у него есть заполненное свойство "location" с названием места, например, город, деревня, остров.

\textbf{Гипотеза.} У войн будет указанно координат меньше (относительно полного числа войн, в процентах), чем у битв, поскольку война - это обычно что-то более протяжённое во времени и пространстве.

\begin{itemize}
\item{Cписок битв с заполненными данными в поле "coordinate location" (См. Рис~\ref{\ref{136121}):}
\begin{lstlisting}[language=SPARQL ]
#List of battle with 'coordinate location' 
#defaultView:Map
SELECT ?battle ?battleLabel ?location
WHERE
{
  ?battle wdt:P31 wd:Q178561. #instance of battle
  ?battle wdt:P625 ?location #display location
  SERVICE wikibase:label { bd:serviceParam wikibase:language "en"}
}
\end{lstlisting}

\href{https://query.wikidata.org/#%23List%20of%20battle%20with%20%27coordinate%20location%27%20%0A%23defaultView%3AMap%0ASELECT%20%3Fbattle%20%3FbattleLabel%20%3Flocation%0AWHERE%0A%7B%0A%20%20%3Fbattle%20wdt%3AP31%20wd%3AQ178561.%20%23instance%20of%20battle%0A%20%20%3Fbattle%20wdt%3AP625%20%3Flocation%20%23display%20location%0A%20%20SERVICE%20wikibase%3Alabel%20%7B%20bd%3AserviceParam%20wikibase%3Alanguage%20%22en%22%7D%0A%7D}{SPARQL-запрос}, 4628 записей.

Вывод по рис.1~\ref{}: большинство битв происходило Европе и 